Nell'ingresso non invertente:
$$\frac{V_1-V_p}{R_{1down}}=\frac{V_p}{R_{2down}}$$
$$\frac{V_1}{R_{1down}}=\frac{V_p}{R_{1down}}+\frac{V_p}{R_{2down}}=V_p \left(\frac{1}{R_{1down}}+\frac{1}{R_{2down}}\right)$$

Nell'ingresso invertente:
$$\frac{V_0-V_n}{R_{2up}}=\frac{V_n}{R_{1up}}$$
$$\frac{V_0}{R_{2up}}=\frac{V_n}{R_{1up}}+\frac{V_n}{R_{2up}}=V_n \left(\frac{1}{R_{1up}}+\frac{1}{R_{2up}}\right)$$

Poichè 
$$V_p=V_n$$

$$\frac{V_1}{R_{1down}} \frac{1}{\left(\frac{1}{R_{1down}}+\frac{1}{R_{2down} }\right)}=\frac{V_0}{R_{2up}} \frac{1}{\left(\frac{1}{R_{1up}}+\frac{1}{R_{2up}} \right)}$$
$$V_0=\frac{R_{2up}}{R_{1down}} \cdot V_1 \frac{\left(\frac{1}{R_{1up}}+\frac{1}{R_{2up} }\right)}{\left(\frac{1}{R_{1down}}+\frac{1}{R_{2down}} \right)}$$
