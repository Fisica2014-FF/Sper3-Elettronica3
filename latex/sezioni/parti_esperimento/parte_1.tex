\subsection{Amplificatore invertente}
Schema amplificatore invertente:
%inserire circuito
Le resistenze sono state scelte in modo da avere guadagno $A=-10 \frac{V}{V}$
$R_1=9.85 \pm \,k\Omega $%mettere errore
$R_2=101.3 \pm \,k\Omega$ %mettere errore
$R_3=56.0 \pm \,\Omega$ %mettere errore

\subsubsection{Calcolo amplificazione}
Dimostrazione che amplificazione in configurazione invertente è data da %inserire dimostrazione

\subsubsection{Analisi}
La stima di A teorica, a partire dalle resistenze misurate è:
$A_{teorica}=$ %inserire valore teorico

Le misure sono state fatte applicando una tensione sinusoidale di frequenza $ f=1 \,kHz$, variando l'ampiezza tra 
$0.2 V_{pp}$ e $4 V_{pp}$.

\begin{grafico} 
 \centering 
 \input{../grafici/risultati/amp_inv.tex} 
 \caption{Curva di trasferimento di un amplificatore invertente} 
 \label{gr:amp_inv.tex} 
\end{grafico}

\begin{tabella}
 \centering
 \begin{center}
\begin{tabulary}{\textwidth}{CCCCCC}
\toprule
$V_{in+}$ (V) & $V_{in-}$ (V) & FS (V) & $V_{out+}$(V) & $V_{out-}$ (V) & FS (V) \\ \midrule 
1.06 & -1.04 & 0.3 & -10.7 & 10.6 & 3 \\ \midrule
0.107 & -0.108 & 0.03 & -1.04 & 1.08 & 0.3 \\ \midrule
0.432 & -0.422 & 0.12 & -4.22 & 4.32 & 1.2 \\ \midrule
0.728 & -0.736 & 0.2 & -7.28 & 7.44 & 2 \\ \midrule
1.36 & -1.38 & 0.4 & -13.8 & 13.9 & 4 \\ \midrule
1.68 & -1.68 & 0.5 & -14.1 & 14.9 & 4 \\ \midrule
1.99 & -1.99 & 0.6 & -14.2 & 14.9 & 4 \\ \midrule
2.09 & -2.09 & 0.6 & -14.2 & 14.9 & 4 \\ \midrule
 \bottomrule
\end{tabulary}
\end{center}
 
 \caption{Dati curva di trasferimento}
 \label{tab:tab_inv.tex}
\end{tabella}

E' stata fatta l'interpolazione lineare pesata dei punti compresi tra 0 e 1.5 V.
%retta interpolazione




\subsection{Amplificatore non invertente}

Schema amplificatore non invertente:
%inserire circuito
Le resistenze sono state scelte in modo da avere guadagno $A=10 \frac{V}{V}$
$R_{1,up}=9.91 \pm \,k\Omega $%mettere errore
$R_{1,down}=9.85 \pm \,k\Omega$ %mettere errore
$R_{2,up}=99.7 \pm \,k\Omega$ %mettere errore
$R_{2,down}=101.3 \pm \,k\Omega$
$R_4=56.0 \pm \,\Omega$

\subsubsection{Calcolo amplificazione}
Dimostrazione che amplificazione in configurazione non invertente è data da %inserire dimostrazione

\subsubsection{Analisi}
La stima di A teorica, a partire dalle resistenze misurate è:
$A_{teorica}=$ %inserire valore teorico

Le misure sono state fatte applicando una tensione sinusoidale di frequenza $ f=1 \,kHz$, variando l'ampiezza tra 
$0.2 V_{pp}$ e $4 V_{pp}$.

\begin{grafico} 
 \centering 
 \input{../grafici/risultati/amp_noninv.tex} 
 \caption{Curva di trasferimento di un amplificatore invertente} 
 \label{gr:amp_noninv.tex} 
\end{grafico}

\begin{tabella}
 \centering
 \begin{center}
\begin{tabulary}{\textwidth}{CCC}
\toprule
$V_{in+}$ (V) & $V_{in-}$ (V) & FS (V) & $V_{out+}$(V) & $V_{out-}$ (V) & FS (V) \\ \midrule 
1.08 & -1.04 & 0.3 & 10.7 & -10.7 & 3 \\ \midrule
0.108 & -0.107 & 0.03 & 1.07 & -1.07 & 0.3 \\ \midrule
0.432 & -0.427 & 0.120 & 4.32 & -4.27 & 1.2 \\ \midrule
0.744 & -0.744 & 0.2 & 7.44 & -7.44 & 2 \\ \midrule 
1.39 & -1.38 & 0.4 & 13.9 & -13.9 & 4 \\ \midrule
1.72 & -1.70 & 0.5 & 14.9 & -14.4 & 4 \\ \midrule
2.04 & -1.99 & 0.6 & 14.7 & -14.2 & 4 \\ \midrule
2.14 & -2.09 & 0.6 & 14.9 & -14.2 & 4 \\ \midrule


\bottomrule
\end{tabulary}
\end{center}
 
 \caption{Dati curva di trasferimento}
 \label{tab:tab_non_inv.tex}
\end{tabella}

E' stata fatta l'interpolazione lineare pesata dei punti compresi tra 0 e 1.5 V.
%retta interpolazione

