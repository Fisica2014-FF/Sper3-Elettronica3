Nota bene: tutti i calcoli sono stati effettuati mantenendo un numero superiore di cifre significative, riducendone il numero solo in sede di presentazione dati.

\subsection{Misure dirette di resistenze}
I valori riportati in \autoref{tab:01_tab_1.tex} (valori in $\Omega$) sono quelli delle misure dirette delle resistenze, prese col multimetro FLUKE 111.

%tabella RES DIRETTE
\begin{tabella}
	\centering
	\begin{center}
\begin{tabulary}{\textwidth}{CCCC}
\toprule
 & R & $\sigma_R$ & $R_{FS}$ \\ \midrule
$R_1$ & 67.8 & 0.4 & 600 \\ \midrule
$R_2$ & 67.9 & 0.4 & 600 \\ \midrule
$R_3$ & 561 & 3 & 600 \\ \midrule
$R_4$ & 1890 & 10 & 6000 \\ \midrule
$R_5$ & 149.8 & 0.8 & 600 \\ \midrule
$R_6$ & 252.0 & 1.3 & 600 \\ 
\bottomrule
\end{tabulary}
\end{center}

	\caption{Misure dirette resistenze}
	\label{tab:01_tab_1.tex}
\end{tabella}


Per stimare gli errori delle misure dirette si \`e usata la formula nell'appendice \textbf{\S\ref{sec:appendice}}.


Per quanto riguarda le resistenze $R_5$ e $R_6$ in serie, da una misurazione diretta effettuata col multimetro FLUKE 111 risulta che $R_{\textrm{S,sper}}= (402 \pm 2 )\Omega$. 

Per il calcolo teorico, \`e stato considerato che $R_{\textrm{S,teor}}=k\cdot(R_5^{(r)} + R_6^{(r)})$: infatti la k \`e costante in misurazioni successive, mantenendo il medesimo fondo scala. Con semplice propagazione degli errori risulta che 
\[\sigma_{R_{\textrm{S,teor}}}=\sqrt{(R_5+R_6)^2\cdot\sigma_k^2+\sigma_{R_5^{(r)}}^2+\sigma_{R_6^{(r)}}^2};\]
\[R_{\textrm{S, teor}}=(402 \pm 2 )\Omega\]
\`E stata calcolata la compatibilit\`a tra le due diverse stime della resistenza, considerando che la loro differenza dovrebbe essere nulla:
\[\Delta R = R_{\textrm{S,teor}} - {R_{\textrm{S,sper}}}= k\cdot(R_{\textrm{S,sper}}^{(r)}-R_5^{(r)}-R_6^{(r)})\]
da cui per propagazione si ricava che 
\[ \sigma_{\Delta R} = 
        \sqrt{ (\Delta R)^2 \sigma_k^2 + 3 \sigma_R^{(r) 2} }\]
e quindi
\[\lambda=\frac{\left|\Delta R - 0\right|}{\sigma_{\Delta R}}=0.5\]





%PARALLELO
%misura
Per il calcolo della resistenza equivalente a $R_5$ e $R_6$ in parallelo, il valore misurato con il multimetro FLUKE 111 \`e $R_{\textrm{P,sper}}= (94 \pm 2) \Omega$.

%teorico
Considerando che $R_{\textrm{P,teor}}=k  \frac{R_5^{(r)} R_6^{(r)}}{R_5^{(r)} + R_6^{(r)}}$ e propagando, riutilizzando la medesima semplificazione sull'errore di scala, si ottiene 
\[\sigma_{\textrm{P,teor}}=\sqrt{\left(\frac{R_5 R_6}{R_5+R_6}\right)^2 \sigma_k^2 + \frac{R_5^4 + R_6^4}{(R_5 + R_6)^4}  \sigma_{R_{\textrm{P,teor}}^{(r)}}^2} .\]
\[R_{\textrm{P,teor}}=(94 \pm 0.5) \Omega\]

%compatibilita'
Ridefinendo \[\Delta R = k \cdot \left(R^{(r)}_{P,sper} - \frac{R_5^{(r)} R_6^{(r)}} {R_5^{(r)} + R_6^{(r)}} \right)\] da cui per propagazione si ricava che 
\[ \sigma_{\Delta R} = 
        \sqrt{ 
               \sigma _k^2 \cdot \Delta R^2 + 
               \sigma _{R^{(r)}}^2 \left(1+\frac{R_5^4 + R_6^4}
                                         {(R_5 + R_6)^4} \right)
        }\]
la compatibilit\`a risulta
\[\lambda=\frac{\left|\Delta R - 0\right|}{\sigma_{\Delta R}}=0.39\]


%%\[\lambda=
%%	\frac{ \left|(R_{\textrm{P,teor}}-R_{\textrm{P,sper}})-0 \right| } { \sigma_{R_{\textrm{P,teor}}}-\sigma_{R_{\textrm{P,sper}}}}=0.14.\] 





 











\subsection{Misura voltamperometrica di una resistenza}
Per misurare una resistenza piccola \`e stato costruito un circuito come in figura. Una prima misura diretta \`e stata effettuata utilizzando il multimetro FLUKE 111, che \`e risultata $R_x=(3.0 \pm 0.1) \Omega$. 

Costruito il circuito, si \`e variata la resistenza di carico e la potenza erogata dal generatore per indagare di quanto fosse la caduta di potenziale al variare della corrente che attraversa R. I dati ottenuti sono riportati in \autoref{tab:02_tab_1.tex}. Il fondo scala \`e di $200mA$ per le correnti e di $600mV$ per le tensioni.

%tabella AMPERE-VOLT
\begin{tabella}
	\centering
	\begin{center}
\begin{tabulary}{\textwidth}{CC}
\toprule
i (mA) & V (mV) \\ \midrule
25.0 & 70.5 \\ \midrule
30.6 & 86.2 \\ \midrule
37.5 & 106.4 \\ \midrule
49.6 & 140.3 \\ \midrule
60.8 & 171.7 \\ \midrule
64.7 & 182.0 \\ \midrule
72.9 & 204.3 \\ \midrule
81.8 & 230.1 \\ \midrule
90.5 & 254.8 \\ \midrule
100.0 & 280.1 \\ 
\bottomrule
\end{tabulary}
\end{center}

	\caption{Misure caduta di potenziale}
	\label{tab:02_tab_1.tex}
\end{tabella}

Nel \autoref{fig:fitlin} sono riportate tali misure esprimendo V in funzione di I, sovrapposte a un fit lineare ottenuto col metodo della massima verosimiglianza.

\begin{grafico}
\centering
\input{../gnuplot/immagini/02_fitlin.tex}
\caption{Fit lineare}
\label{fig:fitlin}
\end{grafico}

I coefficienti della retta interpolante $y=mx+q$ sono:
\[m = (2.798 \pm 0.007) \Omega \] 
\[q = (1.0 \pm 0.5) mV.\]
Utilizzando la formula per la covarianza tra i coefficienti dell'interpolazione lineare, si \`e potuto calcolare
%\[cov(m, c) = -0.000660084 V^4\] %UDM
la correlazione 
\[\rho(m, q) = \frac{\textrm{cov}(m, q)}{\sigma_m \sigma_q}=-0.93\]
e l'errore a posteriori sulla caduta di tensione \`e di $\sigma_V=0.6mV$.

 Questo valore è stato calcolato direttamente dai dati con la formula\footnote{1
 M. Loreti, \textit{Teoria degli Errori e Fondamenti di Statistica}, p. 184
}
\begin{equation}
 \sigma_V = \frac{\sum_i^N \big( y_i-(m x_i+q) \big)^2}{N-2}.
\end{equation}
Risulta un po' alto, ma si vede anche dal \autoref{fig:residui} come un
valore minore di $\sigma_V$ implicherebbe che la maggior parte dei dati sia a più di una sigma dallo zero.

In tale grafico si \`e rappresentata la differenza tra il valore di tensione misurato e quello ricavato teoricamente dalla retta interpolante in corrispondenza del suo valore di corrente.

\begin{grafico}[b]
\centering
\input{../gnuplot/immagini/02_residui.tex}
\caption{Residui}
\label{fig:residui}
\end{grafico}

Una stima della resistenza \`e data dalla pendenza della retta interpolante. Tale retta ha un errore che \`e composizione di un errore sistematico e di uno statistico, infatti si pu\`o scrivere $m=\frac{k_V \big(V_2^{(r)}-V_1^{(r)}\big)}{k_i \big(i_2^{(r)} - i_1^{(r)}\big)}=\frac{k_V}{k_i}m^{(r)}$.
Da una propagazione risulta che l'errore su tale grandezza \`e $\sigma_m=\sqrt{\sigma_{\textrm{m,fit}}^2 + \sigma_{k_V}^2 m^2 + \sigma_{k_i}^2 m^2}$ con $\sigma_{\textrm{m,fit}}$ errore casuale ottenuto attraverso le formule dell'interpolazione.
Risulta che l'incertezza sulla resistenza \`e quasi completamente data dall'errore sistematico. Il risultato finale \`e \[R=(2.80 \pm 0.02) \Omega;\] l'errore percentuale \`e $0.59 \%$.

Si possono confrontare il risultato teorico e quello sperimentale con un calcolo di compatibilit\`a. Dato che sono state usate strumentazioni differenti per le due stime, se ne pu\`o applicare la definizione: 
\[\lambda=\frac{|R_x - R|}{\sqrt{\sigma_{R_x}^2+\sigma_R^2}}=1.7\]
 










\FloatBarrier
\subsection{Resistenze interne degli strumenti di misura}
Attraverso costruzioni di circuiti o misure dirette, si sono stimate le resistenze interne degli strumenti utilizzati.
Per la stima della resistenza interna del generatore si \`e costruito un circuito come in figura e utilizzato il voltmetro AGILENT U1232A con l'amperometro BECKMAN T110B.
Dalle misure risulta che
\begin{align}
V_0 &=(5.01 \pm 0.01 )V \ \textrm{con}\  V_{\textrm{FS}}=6V \\
i   &=(124.9 \pm 0.5) mA \ \textrm{con}\  i_{\textrm{FS}}=200mA \\
V   &=(5.00 \pm 0.01) V \ \textrm{con}\  V_{\textrm{FS}}=6V.
\end{align}

Da uno studio del circuito si ricava la formula $R_G=\frac{V_0-V}{i}$.
Stimandone l'errore, per evitare problemi di correlazione si pu\`o scrivere 
\[R_G=\frac{k_v \big(V_0^{(r)}- V^{(r)}\big)}{i}\] 
da cui propagando: 
\[\sigma_{R_G}=\sqrt{R_G^2 \sigma_{k_V}^2 + \frac{\Big(\sigma_{V^{(r)}}^2 + \sigma_{V_0^{(r)}}^2\Big)}{i^2} + \frac{(V_0-V)^2}{i^4} \sigma_i^2},\] \[R_G= (0.10 \pm 0.01) \Omega\]

Un diverso circuito \`e stato costruito per stimare la resistenza interna dell'AGILENT U1232A utilizzato come voltmetro.
Una misurazione diretta di $R_V$ \`e stata ottenuta utilizzando come ohmetro il BECKMAN T110B: \[R_{\textrm{V, sper}}=(11.2 \pm 0.1) M\Omega\] con fondo scala di $20 M\Omega$. 
Le misure prese a circuito chiuso sono: 
\begin{align}
R_S &= (0.990 \pm 0.005) M\Omega \ \textrm{con}\  R_{\textrm{FS}}=6 M\Omega \\
V_0 &= (5.01 \pm 0.01) V \ \textrm{con}\  V_{\textrm{FS}} = 6 V \\
V &= (4.60 \pm 0.01) V \ \textrm{con}\  V_{\textrm{FS}} = 6 V
\end{align}
Studiando il circuito, si pu\`o dimostrare che $ R_{\textrm{V, teor}} = \frac{R_S V}{V_0 - V}$.
Evidenziando il coefficiente $k_V$ e semplificandolo, si ha $R_{\textrm{V, teor}} = \frac{R_S V^{(r)}}{V_0^{(r)} - V^{(r)}}$ da cui, propagando, si ottiene 
\[\sigma_{R_{\textrm{V,teor}}} = \sqrt{\sigma_{R_S}^2 \left(\frac{V}{V_0 - V} \right)^2 + \sigma_{V^{(r)}}^2 \left(\frac{R_S V_0}{(V_0 - V)^2}\right)^2 + \sigma_{V_0^{(r)}}^2 \left(\frac{R_S V}{(V_0 - V)^2}\right)^2}.\]
\[R_{\textrm{V, teor}} = (11.19 \pm 0.08) M\Omega\]


Per misurare la resistenza interna del BECKMAN T110B, usato come amperometro, si \`e semplicemente effettuato un collegamento con il FLUKE 111 usato come ohmetro. I valori sono riportati in \autoref{tab:03_tab_1.tex}.

%tabella RESISTENZE AMPEROMETRO
\begin{tabella}
	\centering
	\begin{center}
\begin{tabulary}{\textwidth}{CCCC}
\toprule
$I_{FS}$ & $R (\Omega)$ & $\sigma_R (\Omega)$ & $R_{FS} (\Omega)$ \\ \midrule
200 mA & 1002 & 5 & 6000 \\ \midrule
2 mA & 102.1 & 0.5 & 600 \\ \midrule
20 mA & 11.4 & 0.1 & 600 \\ \midrule
200 mA & 1.8 & 0.1 & 600 \\ \midrule
2 A & 1.2 & 0.1 & 600 \\ 
\bottomrule
\end{tabulary}
\end{center}

	\caption{Resistenze dell'amperometro BECKMAN}
	\label{tab:03_tab_1.tex}
\end{tabella}
