\label{sec:appendice}
Si riporta la formula usata per il calcolo dell'incertezza di una misura diretta $A \pm \sigma _A$:
\[  \sigma_A =\sqrt{ \sigma_{\textrm{sist}}^2 + \sigma_{\textrm{stat}}^2}= 0.58 \sqrt{\big(A\cdot\Delta P)^2 + (n_{\textrm{digit}} \cdot \min(\textrm{FS})\big)^2}\]
Infatti gli errori legati alla misurazione sono dovuti sia a errori di scala ($ A= k _A \cdot A^{(r)} $), sia a errori casuali connessi al numero di digit. Per chiarezza di notazione, $\sigma^{(r)}$ \`e considerato errore statistico, mentre con $\sigma$ si intende l'errore totale.

Inoltre, per stimare $\sigma _k$ si \`e utilizzato l'errore percentuale fornito dal costruttore del multimetro, considerando $k$ distribuito uniformemente:
\[\sigma _k = 0.58 \cdot \Delta _P \]
