\'E presentata qua la parte fondamentale del codice in c++ usato per i calcoli numerici. Inoltre è stato usato per i calcoli Mathematica.

% \begin{lstlisting}[language=C++]
% int main() {}
% \end{lstlisting}
\subsection{Parte 1}
\lstinputlisting[language=C++]{../src/opamp_p1/Gain.cpp}
\lstinputlisting[language=C++]{../src/opamp_p1/Graph.h}
\lstinputlisting[language=C++]{../src/opamp_p1/Graph.cpp}
\lstinputlisting[language=C++]{../src/opamp_p1/OpampAnalisys.h}
\lstinputlisting[language=C++]{../src/opamp_p1/OpampAnalisys.cpp}
\lstinputlisting[language=C++]{../src/opamp_p1/readGraph.cpp}

\subsection{Parte 1}
\lstinputlisting[language=C++]{../src/opamp_p2/Gain.cpp}
\lstinputlisting[language=C++]{../src/opamp_p2/Graph.h}
\lstinputlisting[language=C++]{../src/opamp_p2/Graph.cpp}
\lstinputlisting[language=C++]{../src/opamp_p2/OpampAnalysis.h}
\lstinputlisting[language=C++]{../src/opamp_p2/OpampAnalysis.cpp}
\lstinputlisting[language=C++]{../src/opamp_p2/readGraph.cpp}
\lstinputlisting[language=C++]{../src/opamp_p2/AdHocParameters.h}